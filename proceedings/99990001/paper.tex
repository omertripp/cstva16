\documentclass[EPiCempty]{easychair}
\pagestyle{plain}

\begin{document}

\title{An Empirical Understanding of Conflict-Driven Clause-Learning SAT Solvers}
\author{Vijay Ganesh}

\institute{
University of Waterloo, Canada\\
\email{vijay.ganesh@uwaterloo.ca}\\
\url{https://ece.uwaterloo.ca/~vganesh}
}

\maketitle
\paragraph{Abstract.}
Conflict-driven clause-learning (CDCL) SAT solvers have deeply influenced
software engineering and security research over the last two decades, thanks
largely to the fact that these solvers can easily solve real-world constraints
with millions of variables and clauses in them. This phenomenon has puzzled
theoreticians and practitioners alike. It is widely believed that industrial
instances solved efficiently by SAT solvers are highly structured. However,
until recently there was little understanding of the structure of industrial
instances or the way solvers go about exploiting the said structure.
\\\\
In this talk, I will introduce CDCL SAT solvers, and the most important
heuristics that power them. In addition, I will give an answer to the question
of why SAT solvers are efficient on industrial instances. My answer is based on
empirical discoveries my collaborators, students, and I have made via a
rigorous and systematic experimental study of CDCL SAT solvers and industrial
instances. I will conclude with a mathematical model that incorporates our
empirical discoveries.

\paragraph{Biography.} Dr. Vijay Ganesh is an assistant professor at the
University of Waterloo. Prior to that he was a research scientist at MIT, and
completed his PhD in computer science from Stanford University in 2007.
\\\\
Vijay's primary area of research is the theory and practice of automated
reasoning aimed at software engineering, formal methods, security, and
mathematics. In this context he has led the development of many SAT/SMT
solvers, most notably, STP, the Z3 string solver, MapleSAT, and MathCheck. He
has also proved several decidability and complexity results in the context of
theories over string equations and integers. For his research, he won the Early
Researcher Award (ERA) in 2016, an IBM Research Faculty Award in 2015, Google
Research Faculty Awards in 2013 and 2011, and 7 best paper awards/honors at
conferences like ISSTA, SAT, CAV, CADE, SPLC, IJCAI, and DATE (including a
Ten-Year Most Influential Award). His solvers STP and MapleSAT have won
numerous awards at the highly competitive international SMT and SAT solver
competitions. Recently he was invited to the first Heidelberg Laureate Forum in
2013, a gathering where young researchers from around the world were selected
to meet with Turing, Fields and Abel Laureates. 
\end{document}
