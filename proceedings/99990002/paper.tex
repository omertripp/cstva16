\documentclass[EPiCempty]{easychair}
\pagestyle{plain}

\begin{document}

\title{Leveraging Constraint Solvers in Building Reliable Multithreaded Software}
\author{Murali Krishna Ramanathan}

\institute{
Indian Institute of Science, Bangalore, India\\
\email{muralikrishna@csa.iisc.ernet.in}
}

\maketitle
\begin{abstract}
Constraint solvers have been employed successfully in the design of concurrency
debugging tools. In this talk, I will share our experience in using them in the
context of building thread-safe multithreaded libraries. Detecting
concurrency-induced bugs in multithreaded libraries can be challenging due to
the intricacies associated with their manifestation. This includes invocation
of multiple methods, synthesis of inputs to the methods to reach the failing
location, and crafting of thread interleavings that cause the erroneous
behavior. Neither fuzzing-based testing techniques nor over-approximate static
analyses are well positioned to detect such subtle defects while retaining high
accuracy alongside satisfactory coverage.
\\\\
In this talk, I will present a design of a directed, iterative and scalable
testing engine that combines the strengths of static and dynamic analysis to
help synthesize concurrent executions to expose complex concurrency-induced
bugs. Our engine accepts as input the library, its client (either sequential or
concurrent) and a specification of correctness. Then, it iteratively refines
the client to generate an execution which can break the input
specification. Each iteration step includes statically identifying subgoals
towards the goal of failing the specification, generating a plan towards
meeting these goals, and merging of the paths traversed dynamically with the
plan computed statically via constraint solving to generate a new client. The
engine reports full reproduction scenarios, guaranteed to be true, for the bugs
it finds. Our evaluation of the prototype tool based on this design resulted in
detection of 31 real crashes across 10 classes, including classes from the
latest versions of openjdk and google-guava.
\\\\
This is joint work with Malavika Samak and Omer Tripp.
\end{abstract}

\paragraph{Biography.} Murali Krishna Ramanathan is an Assistant Professor in
CSA at Indian Institute of Science, Bangalore. His research interests broadly
span the areas of software engineering, programming languages and scalable
system design. Previously, he was affiliated with Coverity and helped build
program analysis tools that are widely used in the industry. He received a PhD
in Computer Science from Purdue University, USA. 
\end{document}
