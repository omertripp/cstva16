%% \commentNK{ TODO tous ?

%% - Introduction here 
%% }

Concurrent programs are hard to design and implement, especially when
running on multiprocessor architectures. Multiprocessors implement
weak memory models~\cite{AG1996:IEEEC} that allow, for example,
instruction reordering or store buffering. Thus multiprocessors
exhibit more behaviors than Lamport’s \emph{Sequential Consistency}
(SC)~\cite{L79:IEEE}, a theoretical model where the execution of a
program corresponds to an interleaving of the different threads.

Recent years have seen many works on formalization of weak memory
models, both on hardware and software
sides~\cite{%AAS2003:TDPS,
Yang:2005,AM2006:ISCA,SJM2007:PPOPP,BA2008:PLDI,Boudol:2009,AMT2014:TPLS,Abe:2014}.
These formal models give us a way to compute, given a program and a
memory model, the set of concrete executions allowed by the
model. We can determine for example whether all executions of a program 
under a certain model are admitted under SC, and therefore, whether we can
reason about this program using an interleaving semantics.

We propose a constraint solving based technique that allows to
determine all admissible executions of a program under a memory
model. This technique relies on Prolog and CHR (Constraint Handling
Rules)~\cite{Fruhwirth1998}.  A CHR program is a list of rules that act
on a store of constraints, by adding or removing constraints in this
store. The rules are activated if a combination of constraints in the
store match the head of the rule (and satisfy an optional Prolog predicate called the \emph{guard}).  In our
case, constraints are relations between instructions that are
basically related to their order of execution.  Rules are used to
propagate information and to produce, from a given set of relations,
more relations (characterizing the instruction ordering) that were
previously unknown in the execution.

{\bf The contributions} of this paper include a novel CHR-based technique
for detection of admissible behaviors under memory models and a
constraint solver prototype\footnote{Available at \url{http://frederic.loulergue.eu/CHR/wmm.tar.gz}.}
implementing this technique for SC, TSO and PSO
models.  We discuss the benefits and limitations of this approach, and
show how the use of CHR allows us to define our model in a concise and
intuitive way, taking advantage of a well-established specification
and solving mechanism.

{\bf Outline.} %The paper is organized as follows. 
Section~\ref{sect:weakMemoryModels} presents weak memory models vs. SC
model.  Section~\ref{sect:solver} presents the solver, 
explains its functionality and optimizations, 
%the way we implement the rules for a given model and
%compute all admissible executions of a program using the solver, 
and points out some soundness and performance aspects.
Section~\ref{sect:discussion} discusses its benefits and limitations, and gives future work.


%% Local Variables: ***
%% eval: (ispell-change-dictionary "english" nil) ***
%% mode: latex ***
%% eval: (flyspell-buffer) ***
%% End: ***

% LocalWords: CHR Prolog SC TSO
