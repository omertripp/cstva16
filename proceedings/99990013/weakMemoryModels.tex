A \emph{memory model} describes how a program can interact with memory during
its execution. Technical manuals of processors are often vague, and
sometimes erroneous. Formal descriptions are necessary, but must be as
abstract as possible to ease reasoning about them. 

In a parallel context, Lamport~\cite{L79:IEEE} proposed \emph{sequential consistency} (SC)
where the semantics of the parallel composition of two programs is given
by the interleaving of the executions of each program. 
For example, if we consider that two memory locations 
$\mathtt{x}$ and $\mathtt{y}$ initially contain  
$\mathtt{0}$ and that $\mathtt{r_i}$ are processor registers,
the program $p_0$ in Fig.~\ref{fig:programs}a can only give one of
%% \begin{wraptable}{o}{.35\linewidth}
%% \[
%% \begin{array}{c||c}
%%   \begin{array}{l}
%%     \text{Thread~1} \\
%% (1)~    \mathtt{x = 1;} \\
%% (2)~    \mathtt{r_0 = y;} \\
%%   \end{array} &
%%   \begin{array}{l}
%%     \text{Thread~2} \\
%% (3)~    \mathtt{y = 1;} \\
%% (4)~    \mathtt{r_1 = x;} \\
%%   \end{array}
%% \end{array}
%% \]    
%% \end{wraptable}
the three possible final results: either
$\mathtt{r_0 = 0}$ and $\mathtt{r_1 = 1}$, or
$\mathtt{r_0 = 1}$ and $\mathtt{r_1 = 1}$, or
$\mathtt{r_0 = 1}$ and $\mathtt{r_1 = 0}$.
The last result is obtained by the interleaving ($i_{10}$)($i_{11}$)($i_{00}$)($i_{01}$).

\begin{figure}
\begin{subfigure}{.495\textwidth}
\vspace*{2.5mm}
\begin{small}
$\begin{array}{c||c}
  \begin{array}{rl}
%\text{Program } 
p_0{:} &
    \text{Thread~0} \\
&(i_{00})~    \mathtt{x = 1;} \\
&(i_{01})~    \mathtt{r_0 = y;} \\
  \end{array} &
  \begin{array}{l}
    \text{Thread~1} \\
(i_{10})~    \mathtt{y = 1;} \\
(i_{11})~    \mathtt{r_1 = x;} \\
  \end{array}
\end{array}$\\[4mm]
$
\begin{array}{c||c}
  \begin{array}{rl}
%\text{Program } 
p_1{:} & \text{Thread~0} \\
&(i'_{00})~    \mathtt{x = 1;} \\
&(i'_{01})~    \mathtt{fence ;} \\
&(i'_{02})~    \mathtt{r_0 = y;} \\
  \end{array} &
  \begin{array}{l}
\text{Thread~1} \\
(i'_{10})~    \mathtt{y = 1;} \\
(i'_{11})~    \mathtt{fence ;} \\
(i'_{12})~    \mathtt{r_1 = x;} \\
  \end{array}
\end{array}
$
\end{small}
%\caption{Example of programs}
%\label{fig:programs}
~
\end{subfigure}
\begin{subfigure}{.495\textwidth}
\begin{lstlisting}[language=chr]
:- include(sc). % :-include(tso).% for SC or TSO 

program_p0(Vars, [Thread0,Thread1]) :- 
  Vars    = [ x , y ],
  Thread0 = [ (st,x,1), (ld,y,R0) ],
  Thread1 = [ (st,y,1), (ld,x,R1) ].
program_p1(Vars, [Thread0,Thread1]) :-  
  Vars    = [ x , y ],
  Thread0 = [ (st,x,1), f(any,any), (ld,y,R0) ],
  Thread1 = [ (st,y,1), f(any,any), (ld,x,R1) ].

?- program_p0(Vars, P), apply_model(Vars, P).
\end{lstlisting}
%\caption{Implementations of P1 and P2 and\\ example of use of the solver}
%\label{fig:apply-model}
\end{subfigure}
\vspace*{-3mm}
\caption{(a) Two programs $p_0$ and $p_1$,
%Example of programs, 
and (b) their implementations with a solving request.}
\vspace*{-4mm}
\label{fig:programs}
\end{figure}


However, for the sake of performance, no current multiprocessor
provides a sequentially consistent memory model. Models are \emph{relaxed} or
\emph{weak} with respect to SC. In most
processors, it is also possible  to obtain the result
$\mathtt{r_0 = 0}$ and $\mathtt{r_1 = 0}$, not justifiable by an interleaving. To avoid this behavior, 
one can use \emph{memory barriers} or {\em fences} 
that forbid reorderings through them (see
program $p_1$ of Fig.~\ref{fig:programs}a). 

There exist operational models for some weak memory models, for
example~\cite{Boudol:2009}. However this kind of approaches is not
really appropriate to capture the various existing memory models of
processors and programming languages. Most approaches are based on
constraints expressed as partial orders on events. In the case of
memory models, the events of interest are related to memory: \emph{read}
(\emph{load}) from,
and \emph{write} (\emph{store}) 
to a memory location, as well as memory barriers.

Constraint-based approach have already been used to model memory
models, see for example \cite{TVD2010:PLDI}. In this work, we propose a
different approach using the particular case of CHR.
Following~\cite{AMT2014:TPLS}, we use several relations to formalize
memory models. The PO relation, for ``program-order'', simply
expresses that an instruction appears before another one in
the list of instructions defining a thread. The CO relation, for
``coherency-order'', is a per location total order of the write
actions. The RF relation, for ``read-from'', determines which write operation 
assigned the value at a location before this value is read from this location:
%states about the store
%memory access associated to each load: 
it associates a unique write to  each load. The FR relation, for
``from-read'', relates each load operation
with each of the write operations (if any) 
that overwrite the value %at the same location
after it was read by the load.

% Local Variables: ***
% eval: (ispell-change-dictionary "english" nil) ***
% mode: latex ***
% eval: (flyspell-buffer) ***
% End: ***

% LocalWords: CHR Prolog SC
