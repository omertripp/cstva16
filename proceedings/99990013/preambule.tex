\usepackage[utf8]{inputenc} % 
\usepackage{amsmath,amsfonts}
\usepackage{adjustbox}
\usepackage{listings}
\usepackage{courier}
\usepackage{multirow}
%\usepackage{pgfplots}
%\usepackage{pgfplotstable}
\usepackage{hyperref}
%\usepackage{url}
%\usepackage{colortbl}
\usepackage{caption,subcaption,wrapfig}
%\usepackage{aurical}
\usepackage{xspace}
\usepackage{tikz}
%\usepackage{doi}
%\usepackage[numbers]{natbib}
%\usepackage{comment}
%\excludecomment{useless}

%% \usetikzlibrary{matrix,shapes,shapes.multipart,backgrounds,calc,fit,shadows,
%%  positioning}
%%
%% \pgfdeclarelayer{background}
%% \pgfdeclarelayer{foreground}
%% \pgfsetlayers{background,main,foreground}
%% \tikzstyle{plug-in}=[draw, fill=blue!20, text width=4em, 
%%   text centered, minimum height=2em,drop shadow]
%% \tikzstyle{ann} = [above, text width=3em, text centered]
%% \tikzstyle{operation} = [text centered, text width=8em]
%% \tikzstyle{file} = [plug-in, text width=8em, fill=white!20,drop shadow]
%%
%% \definecolor{Gray}{gray}{0.90}
%%
%% \newcommand*\circled[1]{\tikz[baseline=(char.base)]{
%%    \node[shape=circle,draw,inner sep=2pt] (char) {\scriptsize{#1}};}}

\newcommand{\commentAB}[1]{\color{red}((AB: {#1}))\color{black}} 
\newcommand{\commentNK}[1]{\color{orange}((NK: {#1}))\color{black}} 
\newcommand{\commentFL}[1]{\color{blue}((FL: {#1}))\color{black}} 
\newcommand{\commentML}[1]{\color{green}((ML: {#1}))\color{black}} 
\newcommand{\change}[1]{{\color{purple}#1\color{black}}}

% Empty version of comments - uncomment them to measure the real size 
% of the article without comments
%
\renewcommand{\commentAB}[1]{} 
%\renewcommand{\commentNK}[1]{} 
\renewcommand{\commentFL}[1]{} 
\renewcommand{\commentML}[1]{} 
\renewcommand{\change}[1]{#1}

% Some tools and languages:

\newcommand{\FramaC}{\textsc{Frama-C}\xspace}
\newcommand{\framac}{\FramaC}
\newcommand{\PathCrawler}{\textsc{Path\-Crawler}\xspace}
\newcommand{\acsl}{\textsc{acsl}\xspace}
\newcommand{\eacsl}{\textsc{e-acsl}\xspace}
\newcommand{\WP}{\textsc{Wp}\xspace}
\newcommand{\Wp}{\WP}
\newcommand{\Coq}{\textsc{Coq}\xspace}
\newcommand{\WhyThree}{\textsc{Why3}\xspace}
\newcommand{\ZThree}{\textsc{Z3}\xspace}
\newcommand{\CVCThree}{\textsc{CVC3}\xspace}
\newcommand{\CVCFour}{\textsc{CVC4}\xspace}
\newcommand{\Value}{\textsc{Value}\xspace}
\newcommand{\sante}{\textsc{sante}\xspace}
\newcommand{\rte}{\textsc{rte}\xspace}
\newcommand{\boogie}{\textsc{Boogie}\xspace}
\newcommand{\boogaloo}{\textsc{Boogaloo}\xspace}
\newcommand{\symbiotic}{\textsc{Symbiotic}\xspace}
\newcommand{\eve}{\textsc{Eve}\xspace}
\newcommand{\escjava}{\textsc{ESC-Java}\xspace}
\newcommand{\pex}{\textsc{Pex}\xspace}
\newcommand{\sage}{\textsc{Sage}\xspace}
\newcommand{\cute}{\textsc{Cute}\xspace}
\newcommand{\klee}{\textsc{Klee}\xspace}
\newcommand{\smart}{\textsc{Smart}\xspace}
\newcommand{\Occ}{\mathrm{Occ}}
\newcommand{\mappings}{\mathtt{mappings}}
\newcommand{\validpage}{\mathtt{validpage}}
\newcommand{\reff}{\mathtt{ref}}

\newcommand{\zeropage}{\texttt{zero}}%
\newcommand{\datapage}{\texttt{data}}%
\newcommand{\pagetable}{\texttt{pagetable}}%
\newcommand{\pagedirectory}{\texttt{pagedirectory}}%

%% \newcommand{\prettybigletter}{\Fontskrivan\bfseries}
%% \newcommand{\I}{{\prettybigletter (I)}}
%% \newcommand{\G}{{\prettybigletter (G)}}
%% \newcommand{\R}{{\prettybigletter (R)}}

\lstloadlanguages{C}

% --- The following is for a customized pretty-printed and varbatim versions of listing for ACSL annotations. ---

% Warnings: 
% - Do not use other quantified variables than i, j and k
% - Do not add 
%% \lstdefinelanguage{pretty-ACSL}{%
%%   escapechar={},
%%   literate=
%%    {==}{{$==$}}2
%%    {==>}{{$\Rightarrow$}}1
%%    {integer\ i}{{i$\,\in \mathbb{Z}\,$}}4
%%    {integer\ j}{{j$\,\in \mathbb{Z}\,$}}4
%%    {integer\ k}{{k$\,\in \mathbb{Z}\,$}}4
%%    {integer\ m}{{m$\,\in \mathbb{Z}\,$}}4
%%    {integer\ l}{{l$\,\in \mathbb{Z}\,$}}4
%%    {\\forall}{{$\forall$}}1
%%    {\\exists}{{$\exists$}}1
%% %   {integer}{{$\mathbb{Z}$}}1
%%    {real}{{$\mathbb{R}$}}1
%%    {&&}{{$\wedge$}}1
%%    {||}{{$\vee$}}1
%%    {!=}{{$\neq$}}1
%%    {<}{{$<$}}1
%%    {<=}{{$\le~$}}1,
%%   morekeywords={assert,assigns,assumes,axiom,axiomatic,behavior,behaviors,
%%     boolean,breaks,complete,continues,data,decreases,disjoint,ensures,
%%     exit_behavior,ghost,global,inductive,invariant,lemma,logic,loop,
%%     model,predicate,reads,requires,sizeof,strong,struct,terminates,
%%     %returns,
%%     type,union,variant,uchar,byte,typically,\\result,\\old,\\at,\\valid,
%%     \\separated,\\nothing,Pre},
%%   alsoletter={\\},
%%   morecomment=[l]{//}
%% }
%% \lstnewenvironment{pretty-codeACSL}{\lstset{language=pretty-ACSL,stepnumber=0}}{\smallskip}

%% \lstdefinelanguage{ACSL}{%
%%   escapechar={},
%%   literate=,
%%   morekeywords={assert,assigns,assumes,axiom,axiomatic,behavior,behaviors,
%%     boolean,breaks,complete,continues,data,decreases,disjoint,ensures,
%%     exit_behavior,ghost,global,inductive,invariant,lemma,logic,loop,
%%     model,predicate,reads,requires,sizeof,strong,struct,terminates,
%%     %returns,
%%     type,union,variant,uchar,byte,typically,\\result,\\old,\\at,\\valid,
%%     \\separated,\\nothing,Pre},
%%   alsoletter={\\},
%%   morecomment=[l]{//}
%% }
%% \lstnewenvironment{codeACSL}{\lstset{language=ACSL,stepnumber=0}}{\smallskip}



%% \lstdefinestyle{pretty-c}{language={[ANSI]C},%
%%   alsolanguage=pretty-ACSL,%
%%   %commentstyle=\lp@comment,%
%%   moredelim={*[l]{//}},%
%%   %moredelim={*[s]{/*}{*/}},%
%%   %moredelim={**[s]{/*@}{*/}},%
%%   deletecomment={[s]{/*}{*/}},
%%   moredelim={*[l]{//@}},%
%% }

%% \lstdefinestyle{c}{language={[ANSI]C},%
%%   alsolanguage=ACSL,%
%%   %commentstyle=\lp@comment,%
%%   moredelim={*[l]{//}},%
%%   %moredelim={*[s]{/*}{*/}},%
%%   %moredelim={**[s]{/*@}{*/}},%
%%   deletecomment={[s]{/*}{*/}},
%%   moredelim={*[l]{//@}},%
%% }

%% \lstset{language=C,
%%   escapechar=§,
%%   style=pretty-c,
%%   basicstyle=\scriptsize\ttfamily,
%%   numberstyle=\tiny,
%%   numbers=left,
%%   stepnumber=1,
%%   numbersep=5pt,
%%   tab=\rightarrowfill,
%% }

\lstdefinelanguage{chr}{
%%   escapechar=§,
%%   style=pretty-c,
   basicstyle=\scriptsize\ttfamily,
   numberstyle=\scriptsize,
   numbers=left,
   stepnumber=1,
   numbersep=5pt,
   tab=\rightarrowfill,
}
\lstnewenvironment{codeCHR}{\lstset{language=chr,stepnumber=0}}{\smallskip}

\lstMakeShortInline[language=chr,basicstyle=\small\ttfamily]"

%%%%%%%%%%%%%%%%%%%%%%%%%%%%%%%%%%%%%%%%%%%%%%%%%%%
