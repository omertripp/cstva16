\chapter*{Preface}
\setcounter{page}{1}
\pagestyle{plain}


This Conference Proceedings volume contains the written versions of the contributions presented during the Seventh International Workshop on Constraint Solvers in Testing, Verification and Analysis (CSTVA'16). This year the workshop was co-located with the International Symposium on Software Testing and Analysis (ISSTA). 

The venue was Saarbr{\"u}cken, the capital of the Saarland, which is the smallest German federal state. It is located near the French border, halfway between Paris and Frankfurt on the high-speed railway that connects these two cities in less than four hours. Saarbr{\"u}cken offers picturesque attractions and places of historic interest, making it a lovely destination for a hike or a day trip. 

As in previous years, this year too CSTVA provided a platform for researchers and practitioners from the solver community on the one hand, and the programming-languages (PL) and software-engineering (SE) communities on the other hand, to interact, share feedback, exchange views, and form an ambitious and unified vision as to the role of solvers in the PL and SE research domains. The importance of this discussion stems from 
the rapid improvement in expressive power of Boolean
satisfiability (SAT), satisfiability-modulo-theories (SMT) and constraint-programming (CP) solvers throughout the last decade. These impressive advances enable the integration of such solvers into PL/SE techniques and algorithms in new and powerful ways.

The workshop this year covered a wide variety of topics, including use of solvers to reason about weak memory models, implement concolic as well as combinatorial testing efficiently, perform bounded model checking and *invited talks*. It attracted participants from the United States, Germany, *more places*.

The program consisted of three types of contributions: short papers, full papers and invited talks. There were two full-paper presentations, two short-paper presentations and *XXX* invited talks.

We would like to thank all the participants for their contributions to the Workshop program as well as for their contributions to these Proceedings. We extend special thanks to the German participants for their support and hospitality, which allowed all foreign participants to feel more at home. We are also grateful to the invited speakers --- Prof. Vijay Ganesh from Waterloo University, Dr. Julian Dolby from the IBM TJ Watson Research Center and Dr. Murali Krishna Ramanathan from the Indian Institute of Science (IIS), Bangalore --- for their inspiring talks and unwavering acceptance of our invitation.

We look forward to the Eighth International Workshop on Constraint Solvers in Testing, Verification and Analysis. We hope that it will be as interesting and enjoyable as previous CSTVA meetings.

~\bigskip

\noindent
\begin{minipage}[t]{.4\textwidth}
July, 2016
\end{minipage}%
\hfill
\begin{minipage}[t]{.4\textwidth}\flushright
Omer Tripp\\
Christoph M. Wintersteiger
\end{minipage}

\clearpage


\section*{Table of Contents}
  \tocTitle{Design of a Modified Concolic Testing Algorithm with Smaller Constraints}{1}
  \tocAuthors{Yavuz Koroglu and Alper Sen}
  \tocTitle{A CHR-Based Solver for Weak Memory Behaviors}{13}
  \tocAuthors{Allan Blanchard, Nikolai Kosmatov and Frederic Loulergue}
  \tocTitle{A Constraint Solving Problem Towards Unified Combinatorial Interaction\\Testing}{21}
  \tocAuthors{Hanefi Mercan and Cemal Yilmaz}
  \tocTitle{Towards Automated Bounded Model Checking of API Implementations}{28}
  \tocAuthors{Daniel Neville, Andrew Malton, Martin Brain and Daniel Kroening}

\clearpage


\section*{Program Committee}
\noindent
\begin{longtable}{p{0.35\textwidth}p{0.65\textwidth}}
Mathieu Acher & University of Rennes I/INRIA\\
Roberto Bagnara & University of Parma and BUGSENG\\
Martin Brain & University of Oxford, UK\\
Stefano Di Alesio & Certus Centre for Software Verification and Validation,\\ 
  & Simula Research Laboratory\\
Julian Dolby & IBM Thomas J. Watson Research Center\\
Peng Liu & Hong Kong University of Science and Technology\\
Ruben Martins & University of Texas at Austin\\
Corina Pasareanu & CMU/NASA Ames Research Center\\
Markus N. Rabe & University of California, Berkeley\\
Philipp Ruemmer & Uppsala University, Department of Information Technology\\
Philippe Suter & IBM Thomas J. Watson Research Center\\
Omer Tripp & Google\\
Georg Weissenbacher & Vienna University of Technology\\
Christoph M. Wintersteiger & Microsoft Research\\
\end{longtable}

\clearpage
