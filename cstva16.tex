\documentclass{sig-alternate}
\PassOptionsToPackage{hyphens}{url}
\usepackage{hyperref}
\begin{document}

\title{Workshop on Constraints in Software Testing, Verification, and
  Analysis (CSTVA'14)}

\numberofauthors{2}

\author{
\alignauthor
Omer Tripp\\
       \affaddr{EMPTY}\\
       \affaddr{EMPTY}\\
       \affaddr{EMPTY}\\
       \email{EMPTY}\\
% 2nd. author
\alignauthor
Christoph M. Wintersteiger\\
       \affaddr{Microsoft Research}\\
       \affaddr{21 Station Road}\\
       \affaddr{Cambridge, CB1 2FB, United Kingdom}\\
       \email{cwinter@microsoft.com}\\
}
\date{October 23, 2013}

\maketitle
\toappear{}

\section{Main contact}
\vspace{0.2cm}
Vijay Ganesh (see above for contact details).

\section{Workshop length}
\vspace{0.2cm}
One day

\section{Workshop dates}
\vspace{0.2cm}
Preferably before the start of the ICSE conference.

\section{Workshop Open or Closed}
\vspace{0.2cm}
Open.

\section{Min/Max Number of Participants}
\vspace{0.2cm} 

Minimum: 15, and Maximum: 50. We may not organize the workshop if
there are fewer than 15 participants.
\vspace{0.2cm}

\begin{abstract}
\vspace{0.2cm}
The last decade has seen a revolutionary improvement in the efficiency
and expressive power of Boolean SAT, SMT and Constraint Programming
(CSP/CP) solvers, with a consequent impact on all manner of software
engineering (SE) applications and research programs. A prime example
of this is the rapid development and adoption of solver-based
symbolic-execution techniques in myriad applications such as test
generation, security, and analysis. Despite this increasing use of
solvers, there are few venues that are solely dedicated to bringing
together the broader SE and solver communities in one place. This
workshop at the flagship SE conference is designed precisely to fill
this gap. The aim of the workshop is to highlight use of solvers in
novel applications, new solver features, and encourage their use in
fresh solutions to long-standing SE problems. Furthermore, as current
users become more sophisticated they demand richer solver interfaces
that are extensible, tunable and programmable. Designing such
interfaces requires deeper interaction between solver developers and
users. Hence, this workshop will also act as a venue for feedback from
users to developers.

\end{abstract}

\section{Workshop theme and goals}
\vspace{0.2cm} Recent years have seen an increasing interest in the
application of constraint solving techniques to testing, verification
and analysis of software systems. The reason for this interest is the
dramatic improvement in efficiency and expressive power of Boolean
SAT, SMT and CP solvers, thus making it considerably easier to build
and maintain software engineering (SE) applications. Significant
number of constraint-based techniques have been proposed and
investigated in model-based testing, code-based testing,
property-oriented testing, statistical testing, equivalence checking,
model checking, fault localization, verification, and program
analysis.  What is even more interesting is that as solvers become
more efficient and expressive, newer applications are being developed
in diverse areas such as security, synthesis, type systems and
software product lines. In the past, solvers were largely used as
black boxes. However, more recently a central idea in solver-based SE
applications is a deep integration of solver and application. The
benefits of deep integration over a black box approach are many. An
example benefit of deeper integration is that it can enable
applications to better utilize the adaptive features of modern
solvers. Often the resulting SE techniques are far more powerful and
effective than before.

\vspace{0.2cm} It is within this context of increasing use of solvers
in SE, especially through deep integration of extensible solvers with
applications, that we propose a new workshop on solvers and software
engineering at ICSE, the flagship SE conference. The goal of the
workshop is to bring together in one place the largely distinct
communities of solver and broader SE researchers. We expect a very
intense and positive cross-community interaction between the members
of the two communities. Such interaction could lead to completely new
approaches to long-standing SE problems, and current users of solvers
can discuss their experience and provide rapid feedback to solver
developers. The workshop could also influence new research directions
into easily extensible, tunable and programmable solvers. Finally, the
workshop can help showcase the breadth and depth of the influence of
solvers in SE research.

\vspace{0.2cm} A prime example of how interaction between solver
researchers and the SE community can lead to impactful research is
``symbolic execution and its application to testing, verification and
analysis''. Symbolic execution for software testing was proposed by
Lori Clarke and J.C. King in the 1970's. However, it didn't really
pick up until nearly thirty years later in the mid-2000's, in large
measure due to dramatic improvement in solver technology. Thanks to
SAT solvers like MiniSAT, Lingeling, ManySAT, and SMT solvers like Z3,
Yices, CVC4, MathSAT, STP, and HAMPI, a variety of symbolic-execution
engines have been developed. Examples of the most successful such
tools include SAGE, PathCrawler, KLEE (and its predecessor EXE),
BitBlaze, WebBlaze, S2E, Java PathFinder and KINT.

\vspace{0.2cm} Another example of deep integration of solver and
application is the recently proposed model-checking algorithm IC3.  In
this approach, not only is a solver used by the model- checker, but
also the design of the IC3 algorithm itself is influenced by ideas
from SAT solving.

\vspace{0.2cm} A third example comes from the development of
specialised CP techniques for constraints on floating-point numbers
which could not be accurately resolved using the mathematical tools
developed for real numbers. This is a significant recent advance in
solver technology with huge potential for impact in automated testing
and verification of numerical software.

\vspace{0.2cm} We believe that the success we have seen of the
combination of solvers and above-mentioned SE fields can be replicated
in other areas of SE. It is with this goal of increasing the awareness
of solvers in the broader SE community, and increasing interaction
between solver and SE researchers that the workshop is being proposed.

\vspace{0.2cm} The workshop will focus on a broad range of topics
where solvers have already made an impact, e.g., symbolic-execution
based testing, as well as newer ones where their use is still nascent,
e.g., synthesis and software product lines.

\subsection*{Topics for the Workshop}
\vspace{0.2cm}
\noindent{The topics for the workshop include, but are not limited to,
  the following:}

\begin{itemize}
\item Constraint-based analysis of programs and models
\item Constraint-based test input generation
\item SMT solvers for testing, verification, analysis, and synthesis
\item SMT solvers and their applications in computer security
\item Programmable SMT solvers
\item Combinations of constraint solvers
\item Constraint programming and software engineering
\item Solvers and software product lines
\item Solvers and fault localization
\end{itemize}

\section{Relevance to the field of SE}
\vspace{0.2cm} The idea of capturing software behavior through logic
constraints is an old but very important idea that has always played a
role in SE.  However, in the past SE researchers didn't have access to
efficient solvers, and instead had to develop their own solutions to
process the constraints generated by their applications. All that
changed in the last 15 years. Armed with increasingly powerful SAT/SMT
solvers, researchers tackled many hard SE problems. As mentioned
above, many areas of SE have been transformed thanks to solvers and we
expect that the broader SE community will benefit greatly from
continued improvement in solver technology, and deeper integration of
solver and application.

\vspace{0.2cm} \sloppypar{In addition to the above-mentioned
  applications that have been impacted by solvers, new and emerging
  areas where solvers are used heavily include software product lines,
  constraint-based program analysis and fault localization. These
  application domains are quite distinct from each other, and yet they
  all rely on powerful solvers. For example, in software product lines
  (SPL) research the goal is to show that the user-selected features
  and configuration can lead to a valid product, that does not violate
  important safety and security properties. On the surface, the SPL
  research area seems very different from the problem of fault
  localization, where the problem is to isolate the root cause(s) of
  an error given a program and an error-revealing input. In the
  constraint-based program analysis field, the paradigm is to express
  the progam analysis problem as constraints in declarative languages
  such as Prolog. These constraints are then interpreted by powerful
  solvers to perform analysis on input programs. In the past,
  researchers in each of these application areas had to develop
  individual specialised solutions to address their problems. As
  already highlighted in the above Section, thanks to the advent of
  efficient and expressive solvers, a new class of systematic methods
  is being explored to solve these knotty problems.}

\section{Generating discussion}
\vspace{0.2cm} We expect a very robust attendance at the workshop. We
plan to organize a panel discussion on the future of solvers in SE
that will bring together 2 leading solver developers and 2 leading
users to present their views on how we can build solvers that better
serve SE research needs. We expect to hold the panel towards the end
of the workshop. We also hope to invite a leading SE researcher to
give a keynote address on his/her experience in using solvers.

\vspace{0.2cm} We are also considering the possibility of
presentation-only papers (e.g., tool papers) that can generate a lot
of interest. Such papers can be shorter, say, 6 pages and will not be
included in any proceedings.

%% \section{Limits on no. of participants}

%% The workshop will not be held if there are fewer than 15 participants
%% and applications to participate will be refused if there are more than
%% 50.

\section{Preliminary web site}
\vspace{0.2cm} \url{https://ece.uwaterloo.ca/~vganesh/cstva14.html}

\section{Program Committee}
\vspace{0.2cm} We are fortunate that many of the leading researchers
in the field of SAT/SMT solvers, CP and SE applications have agreed to
be part of the program committee of the proposed workshop. We believe
that this reflects the increasing recognition of the fact that the
broader community needs a workshop or conference where developers of
different constraint solving techniques and their users can meet and
exchange ideas.

\vspace{0.2cm}
The following people have all already agreed to be on the Program
Committee.

\begin{enumerate}
\item Vijay Ganesh, University of Waterloo, Canada
\item Nicky Williams, CEA LIST, France
\item Kapil Vaswani, Microsoft Research, India
\item Aditya Nori, Microsoft Research, India
\item Rupak Majumdar, Max Planck Institute for Software Systems, Germany
\item Koushik Sen, University of California, Berkeley, USA
\item Frank Tip, University of Waterloo, Canada
\item Joxan Jaffar, National University of Singapore, Singapore
\item Nikolaj Bjorner, Microsoft Research, USA
\item Leonardo DeMoura, Microsoft Research, USA
\item Cristian Cadar, Imperial College London, UK
\item Arnaud Gotlieb, Simula, Norway and INRIA, France
\item Frederic Dadeau, FEMTO-ST/INRIA, France
\item Krzysztof Czarnecki, University of Waterloo, USA
\item Julian Dolby, IBM TJ Watson Center, USA
\item Ofer Strichman, Technion, Israel
\item Fran{\c c}ois Bobot, CEA LIST, France
\item S{\'e}bastien Bardin, CEA LIST, France
\item Sylvain Conchon, Universit{\'e} Paris Sud, France
\item Emina Torlak, University of California, Berkeley, USA
\item Daniel LeBerre, Universit{\'e} d'Artois, France
\item Cesare Tinelli, University of Iowa, USA
\item Chris Wintersteiger, Microsoft Research, UK
\item Patrick Heymans, University of Namur, Belgium
\item Marsha Chechik, University of Toronto, Canada
\item Xiangyu Zhang, Purdue University, USA
\end{enumerate}

\section{Participant solicitation}
\vspace{0.2cm} Participants will be solicited through a mailshot of
the Call For Papers and a poster. This should reach a wide audience
thanks to the diversity of the program committee.  The workshop will
be open to all participants until the maximum number of participants
is reached.

\section{Proceedings}
\vspace{0.2cm} Accepted papers (except presentation-only papers) will
be published in the workshop proceedings.  The ideal number of papers
is around 12 and expected length is 10 pages with 1-2 additional pages
for references in standard ICSE format.

\section{Expected attendance}
\vspace{0.2cm} This proposed workshop will be the 6th of the CSTVA
(Constraints in Software Testing, Verification and Analysis) workshop
series, held for the last few years alongside ICST (IEEE International
Conference on Software Testing).

\vspace{0.2cm} \sloppypar{The first CSTVA meeting was held alongside
  the CP conference (Principles and Practice of Constraint
  Programming) in Nantes, France, in 2006 and attracted more than 25
  participants. It also featured an invited presentation by Andy King
  (University of Kent). See
  \url{http://www.irisa.fr/manifestations/2006/CSTVA06/} for more
  details.

\vspace{0.2cm} The second edition took place in conjunction with the
ICST 2010 in Paris, France, attracting more than 30 participants. It
also featured an invited presentation by Peli de Halleux (senior
researcher at Microsoft Research).  See
\url{http://www.st.cs.uni-saarland.de/cstva10} for more details.

\vspace{0.2cm} A third meeting at ICST 2011 in Berlin, Germany, also
attracted a similar number of attendees, and featured an invited
presentation of Patrice Godefroid (principal researcher at Microsoft
Research).  See \url{http://www.st.cs.uni-saarland.de/cstva11} for
more details.

\vspace{0.2cm} A fourth edition was held at ICST 2012 in Montreal,
Canada which attracted close to 40 participants. It also featured a
keynote presentation by Vijay Ganesh (an assistant professor at the
University of Waterloo, and a co-organizer of the proposed workshop.)
See \url{http://srg.doc.ic.ac.uk/cstva12/} for more details.

\vspace{0.2cm} A fifth edition was held at ICST 2013 in Luxembourg
which attracted more than 40 participants. It featured a keynote
address by Sarfraz Kurshid (University of Texas, Austin). See
\url{http://cstva2013.univ-fcomte.fr/index.php?aim} for more details.
}

\vspace{0.2cm} Given the success of the previous editions of this
workshop we believe that the current version we are proposing will
also be successful. We believe that by co-locating with ICSE and
broadening the scope of the workshop to go beyond testing,
verification and analysis, and include synthesis, security and
software product lines etc., will increase the chances of success of
the workshop dramatically. We are hopeful that we will attract more
than 40 participants.

\vspace{0.2cm} As mentioned before, we informally discussed the idea
of organizing such an ICSE workshop with many leading SE researchers,
and found that most were very enthusiastic about it. Most of the
researchers we contacted gladly accepted our invitation to join the
program committee.

\section{Logistic constraints}
\vspace{0.2cm} The room should be covered by wifi if possible, in case
of videos or on-line demonstrations.

\section{Organizer background}
\vspace{0.2cm}
\begin{enumerate}
\item {\bf Vijay Ganesh} is an assistant professor at the University
  of Waterloo, Ontario, Canada. His primary research interests are
  SAT/SMT solvers and their applications to testing, formal methods
  and SE in general. He has developed leading SMT solvers, and also
  worked on their application to testing and formal verification. His
  award-winning solvers STP and HAMPI are currently being used in more
  than 100 research projects.

  On the organizational front, Vijay organized one of the first summer
  schools on SAT/SMT solvers in 2011 at the Massachusetts Institute of
  Technology. The summer school featured 36 lectures by leading solver
  developers and their users over a period of 6 days. The total number
  of participants exceeded 230, and included participants from all
  over the world. The profile of the pariticipants ranged from
  theoreticians interested in understanding why SAT solvers work so
  well, to hackers who wanted to use solvers for a variety of security
  applications. This summer school has started trend, and now it is
  held every year alongside the SAT or CAV conferences. More details
  can be found at:
  \url{http://people.csail.mit.edu/vganesh/summerschool/index.html}. Vijay
  has also served on the program committee of many conferences and
  workshops including FroCos, IEEE NCA, and SMT (usually co-held with
  CAV).

\item {\bf Nicky Williams} is a leading researcher in symbolic
  execution based methods for testing and verification. She is a
  research engineer in the CEA LIST Software Reliability Lab, which
  works closely with industry on the application of new techniques to
  the verification and validation of software systems. Nicky initiated
  the development of the CLP-based PathCrawler test generation tool
  and has worked for several years on its extension and application to
  different industrial problems.

  Nicky has been a member of the program committee for several
  conferences and workshops in the domain, such as ICTSS, QSIC, AFADL
  and several previous editions of the CSTVA workshop and this year
  decided to participate for the first time in the organisation of
  this workshop.
\end{enumerate}

\end{document}
